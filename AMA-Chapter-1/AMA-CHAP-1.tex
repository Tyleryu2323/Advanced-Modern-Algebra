% --------------------------------------------------------------
% This is all preamble stuff that you don't have to worry about.
% Head down to where it says "Start here"
% --------------------------------------------------------------
 
\documentclass[12pt]{article}
 
\usepackage[margin=1in]{geometry} 
\usepackage{amsmath,amsthm,amssymb,scrextend}
\newcommand{\N}{\mathbb{N}}
\newcommand{\Z}{\mathbb{Z}}
 
\newenvironment{theorem}[2][Theorem]{\begin{trivlist}
\item[\hskip \labelsep {\bfseries #1}\hskip \labelsep {\bfseries #2.}]}{\end{trivlist}}
\newenvironment{lemma}[2][Lemma]{\begin{trivlist}
\item[\hskip \labelsep {\bfseries #1}\hskip \labelsep {\bfseries #2.}]}{\end{trivlist}}
\newenvironment{exercise}[2][Exercise]{\begin{trivlist}
\item[\hskip \labelsep {\bfseries #1}\hskip \labelsep {\bfseries #2.}]}
{\end{trivlist}}

\newenvironment{reflection}[2][Reflection]{\begin{trivlist}
\item[\hskip \labelsep {\bfseries #1}\hskip \labelsep {\bfseries #2.}]}{\end{trivlist}}
\newenvironment{proposition}[2][Proposition]{\begin{trivlist}
\item[\hskip \labelsep {\bfseries #1}\hskip \labelsep {\bfseries #2.}]}{\end{trivlist}}
\newenvironment{corollary}[2][Corollary]{\begin{trivlist}
\item[\hskip \labelsep {\bfseries #1}\hskip \labelsep {\bfseries #2.}]}{\end{trivlist}}
 
 
\newenvironment{hint}[2][Hint]{\begin{trivlist}
    \item[\hskip \labelsep {\bfseries #1}\hskip \labelsep {\bfseries #2.}]}
    {\end{trivlist}}
\usepackage[margin=1in]{geometry} 

\usepackage{fancyhdr}
\pagestyle{fancy}

\newcommand{\cont}{\subseteq}
\usepackage{tikz}
\usepackage{pgfplots}
\usepackage{amsmath}
\usepackage[mathscr]{euscript}
\let\euscr\mathscr \let\mathscr\relax% just so we can load this and rsfs
\usepackage[scr]{rsfso}
\usepackage{amsthm}
\usepackage{amssymb}
\usepackage{multicol}
\usepackage[colorlinks=true, pdfstartview=FitV, linkcolor=blue,
citecolor=blue, urlcolor=blue]{hyperref}

\DeclareMathOperator{\arcsec}{arcsec}
\DeclareMathOperator{\arccot}{arccot}
\DeclareMathOperator{\arccsc}{arccsc}
\newcommand{\ddx}{\frac{d}{dx}}
\newcommand{\dfdx}{\frac{df}{dx}}
\newcommand{\ddxp}[1]{\frac{d}{dx}\left( #1 \right)}
\newcommand{\dydx}{\frac{dy}{dx}}
\let\ds\displaystyle
\newcommand{\intx}[1]{\int #1 \, dx}
\newcommand{\intt}[1]{\int #1 \, dt}
\newcommand{\defint}[3]{\int_{#1}^{#2} #3 \, dx}
\newcommand{\imp}{\Rightarrow}
\newcommand{\un}{\cup}
\newcommand{\inter}{\cap}
\newcommand{\ps}{\mathscr{P}}
\newcommand{\set}[1]{\left\{ #1 \right\}}
\newtheorem*{sol}{Solution}
\newtheorem*{claim}{Claim}
\newtheorem{problem}{Problem}
\newtheorem{solution}{Solution:}
\usepackage{setspace}
\begin{document}
 
\onehalfspacing
 
\title{Advanced Modern Algebra second edition} 
\author{Selected Solutions\\ 
Chapter 1: Groups I}
 
\maketitle

\section*{1.1.  Classical Formulas}


\begin{exercise}{1.1} 
    Given $M, N \in \mathbb{C}$, prove that there exists $g, h \in \mathbb{C}$ with $g+h=M$ and $gh=N$.
\end{exercise}

\begin{proof}
    Consider the quadratic equation $x^2-Mx+N=0$ and apply the quadratic formula, we have two roots ${r_1 = \frac{-M+ \sqrt{M^2-4N}}2} $ and ${r_2 = \frac{-M - \sqrt{M^2-4N}}2}$. Notice that ${r_1+r_2=-M}$ and ${r_1r_2=N}$. Then we see that $-r_1, -r_2 \in \mathbb{C}$ that satisfies the relation.
\end{proof}


\begin{exercise}{1.3}
    \begin{enumerate}
        \item[(i)] Find the complex roots of $f(x)=x^3-3x+1$.
        \item[(ii)] Find the complex roots of $f(x)=x^4-2x^2+8x-3$.
    \end{enumerate}
\end{exercise}
 




 \begin{exercise}{1.4}
    Show that the quadratic formula does not hold for $f(x)=ax^2+bx+c$ if we view the coefficients $a,b,c$ as lying in the integers mod 2.
 \end{exercise}





\section*{1.2.  Permutations}

 \begin{exercise}{1.5}

 Give an example of functions $f: X\rightarrow Y$ and $g: Y\rightarrow X$ such that $gf = 1_{X}$ and $fg \neq 1_Y$.
 \end{exercise} 

 \begin{proof}
    Consider $f:\mathbb{Z}\rightarrow\mathbb{Z}, g:\mathbb{Z}\rightarrow\mathbb{Z}$ where $f(x) = -x, g(x) = |x|$. Then we see that ${gf(x)=|-x|=x}$,  $\forall x\in \mathbb{Z}$ while ${fg(1)=-|1|=-1}$.
 \end{proof}

 \begin{exercise}{1.6}

Prove that the composition of functions is associative: if $X\xrightarrow{f}Y\xrightarrow{g}Z\xrightarrow{h}W$, then \begin{align*}h(gf)=(hg)f.\end{align*}
 \end{exercise}
\begin{proof}
     $h(gf) (x) = h(g(f(x)) = (gh)f(x)$ and 
\end{proof}

\begin{exercise}{1.7}

Prove that the composite of two injections is an injection, and that the composite of two surjections is a surjection. Conclude that the composite of two bijections is a bijection.
\end{exercise}

\begin{exercise}{1.8 (Pigeonhole Principle)}
    \begin{enumerate}

        \item[(i)] Let $f:X\rightarrow X$ be a function, where $X$ is a finite set. Prove equavalence of the following statements.
        \begin{enumerate}
            \item[(a)] $f$ is an injection.
            \item[(b)] $f$ is a bijection.
            \item[(c)] $f$ is a surjection.
        \end{enumerate}

        \item[(ii)] Prove that no two of the statements in (i) are equivalent when $X$ is an infinite set.
        \item[(iii)] Suppose there are 501 pigeons, each sitting in some pigeonhole. If there are only 500 pifeonholes, prove that there is a hole containing more than one pigeon.
    \end{enumerate}
\end{exercise}



\begin{exercise}{1.9}

    Let $Y$ be a subset of a finite set $X$, and let $f:Y\rightarrow X$ be an injection. Prove that there is a permutation $\alpha \in S_X$ with $\alpha |Y =f$.

    
\end{exercise}


\begin{exercise}{1.10}

    Find sgn$(\alpha)$ and $\alpha^{-1}$, where 
    \begin{align*}
        \alpha = \begin{pmatrix}
            1&2&3&4&5&6&7&8&9\\9&8&7&6&5&4&3&2&1
        \end{pmatrix}.
    \end{align*}
    
\end{exercise}


\begin{exercise}{1.11}

    If $\alpha\in S_n$, prove that sgn$(\alpha^{-1}) =$ sgn$(\alpha)$.
    
\end{exercise}



\begin{exercise}{1.12}

    If $1\leq r \leq n$, show that there are 
    \begin{align*}
        \frac{1}r [n(n-1)...(n-r+1)]
    \end{align*} 
    $r$-cycles in $S_n$.

    \begin{hint}
        if 
    \end{hint}

\end{exercise}







\end{document}